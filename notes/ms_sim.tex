\documentclass[10pt,fleqn]{article}
% \usepackage[journal=rsc]{chemstyle}
% \usepackage{mhchem}
\usepackage{amsmath}
\usepackage{amssymb}
\usepackage{amsfonts}
\usepackage{esint}
\usepackage{bbm}
\usepackage{amscd}
\usepackage{picinpar}
\usepackage{graphicx}
\usepackage{tikz}
\usepackage{tikz-3dplot}
\usepackage{indentfirst}
\usepackage{wrapfig}
\usepackage{units}
\usepackage{textcomp}
\usepackage[utf8x]{inputenc}
% \usepackage{feyn}
\usepackage{feynmp}
\usepackage{xkeyval}
\usepackage{xargs}
\usepackage{verbatim}
\usepackage{pgfplots}
\usepackage{hyperref}
\usetikzlibrary{
  arrows,
  calc,
  decorations.pathmorphing,
  decorations.pathreplacing,
  decorations.markings,
  fadings,
  positioning,
  shapes
}

\DeclareGraphicsRule{*}{mps}{*}{}
\newcommand{\ud}{\mathrm{d}}
\newcommand{\ue}{\mathrm{e}}
\newcommand{\ui}{\mathrm{i}}
\newcommand{\res}{\mathrm{Res}}
\newcommand{\Tr}{\mathrm{Tr}}
\newcommand{\dsum}{\displaystyle\sum}
\newcommand{\dprod}{\displaystyle\prod}
\newcommand{\dlim}{\displaystyle\lim}
\newcommand{\dint}{\displaystyle\int}
\newcommand{\fsno}[1]{{\!\not\!{#1}}}
\newcommand{\eqar}[1]
{
  \begin{align*}
    #1
  \end{align*}
}
\newcommand{\texp}[2]{\ensuremath{{#1}\times10^{#2}}}
\newcommand{\dexp}[2]{\ensuremath{{#1}\cdot10^{#2}}}
\newcommand{\eval}[2]{{\left.{#1}\right|_{#2}}}
\newcommand{\paren}[1]{{\left({#1}\right)}}
\newcommand{\lparen}[1]{{\left({#1}\right.}}
\newcommand{\rparen}[1]{{\left.{#1}\right)}}
\newcommand{\abs}[1]{{\left|{#1}\right|}}
\newcommand{\sqr}[1]{{\left[{#1}\right]}}
\newcommand{\crly}[1]{{\left\{{#1}\right\}}}
\newcommand{\angl}[1]{{\left\langle{#1}\right\rangle}}
\newcommand{\tpdiff}[4][{}]{{\paren{\frac{\partial^{#1} {#2}}{\partial {#3}{}^{#1}}}_{#4}}}
\newcommand{\tpsdiff}[4][{}]{{\paren{\frac{\partial^{#1}}{\partial {#3}{}^{#1}}{#2}}_{#4}}}
\newcommand{\pdiff}[3][{}]{{\frac{\partial^{#1} {#2}}{\partial {#3}{}^{#1}}}}
\newcommand{\diff}[3][{}]{{\frac{\ud^{#1} {#2}}{\ud {#3}{}^{#1}}}}
\newcommand{\psdiff}[3][{}]{{\frac{\partial^{#1}}{\partial {#3}{}^{#1}} {#2}}}
\newcommand{\sdiff}[3][{}]{{\frac{\ud^{#1}}{\ud {#3}{}^{#1}} {#2}}}
\newcommand{\tpddiff}[4][{}]{{\left(\dfrac{\partial^{#1} {#2}}{\partial {#3}{}^{#1}}\right)_{#4}}}
\newcommand{\tpsddiff}[4][{}]{{\paren{\dfrac{\partial^{#1}}{\partial {#3}{}^{#1}}{#2}}_{#4}}}
\newcommand{\pddiff}[3][{}]{{\dfrac{\partial^{#1} {#2}}{\partial {#3}{}^{#1}}}}
\newcommand{\ddiff}[3][{}]{{\dfrac{\ud^{#1} {#2}}{\ud {#3}{}^{#1}}}}
\newcommand{\psddiff}[3][{}]{{\frac{\partial^{#1}}{\partial{}^{#1} {#3}} {#2}}}
\newcommand{\sddiff}[3][{}]{{\frac{\ud^{#1}}{\ud {#3}{}^{#1}} {#2}}}
\usepackage{fancyhdr}
\usepackage{multirow}
\usepackage{fontenc}
% \usepackage{tipa}
\usepackage{ulem}
\usepackage{color}
\usepackage{cancel}
\newcommand{\hcancel}[2][black]{\setbox0=\hbox{#2}%
  \rlap{\raisebox{.45\ht0}{\textcolor{#1}{\rule{\wd0}{1pt}}}}#2}
\pagestyle{fancy}
\setlength{\headheight}{67pt}
\fancyhead{}
\fancyfoot{}
\fancyfoot[C]{\thepage}
\fancyhead[R]{}
\renewcommand{\footruleskip}{0pt}
\renewcommand{\headrulewidth}{0.4pt}
\renewcommand{\footrulewidth}{0pt}

\newcommand\pgfmathsinandcos[3]{%
  \pgfmathsetmacro#1{sin(#3)}%
  \pgfmathsetmacro#2{cos(#3)}%
}
\newcommand\LongitudePlane[3][current plane]{%
  \pgfmathsinandcos\sinEl\cosEl{#2} % elevation
  \pgfmathsinandcos\sint\cost{#3} % azimuth
  \tikzset{#1/.estyle={cm={\cost,\sint*\sinEl,0,\cosEl,(0,0)}}}
}
\newcommand\LatitudePlane[3][current plane]{%
  \pgfmathsinandcos\sinEl\cosEl{#2} % elevation
  \pgfmathsinandcos\sint\cost{#3} % latitude
  \pgfmathsetmacro\yshift{\cosEl*\sint}
  \tikzset{#1/.estyle={cm={\cost,0,0,\cost*\sinEl,(0,\yshift)}}} %
}
\newcommand\DrawLongitudeCircle[2][1]{
  \LongitudePlane{\angEl}{#2}
  \tikzset{current plane/.prefix style={scale=#1}}
  % angle of "visibility"
  \pgfmathsetmacro\angVis{atan(sin(#2)*cos(\angEl)/sin(\angEl))} %
  \draw[current plane] (\angVis:1) arc (\angVis:\angVis+180:1);
  \draw[current plane,dashed] (\angVis-180:1) arc (\angVis-180:\angVis:1);
}
\newcommand\DrawLatitudeCircleArrow[2][1]{
  \LatitudePlane{\angEl}{#2}
  \tikzset{current plane/.prefix style={scale=#1}}
  \pgfmathsetmacro\sinVis{sin(#2)/cos(#2)*sin(\angEl)/cos(\angEl)}
  % angle of "visibility"
  \pgfmathsetmacro\angVis{asin(min(1,max(\sinVis,-1)))}
  \draw[current plane,decoration={markings, mark=at position 0.6 with {\arrow{<}}},postaction={decorate},line width=.6mm] (\angVis:1) arc (\angVis:-\angVis-180:1);
  \draw[current plane,dashed,line width=.6mm] (180-\angVis:1) arc (180-\angVis:\angVis:1);
}
\newcommand\DrawLatitudeCircle[2][1]{
  \LatitudePlane{\angEl}{#2}
  \tikzset{current plane/.prefix style={scale=#1}}
  \pgfmathsetmacro\sinVis{sin(#2)/cos(#2)*sin(\angEl)/cos(\angEl)}
  % angle of "visibility"
  \pgfmathsetmacro\angVis{asin(min(1,max(\sinVis,-1)))}
  \draw[current plane] (\angVis:1) arc (\angVis:-\angVis-180:1);
  \draw[current plane,dashed] (180-\angVis:1) arc (180-\angVis:\angVis:1);
}
\newcommand\coil[1]{
  {\rh * cos(\t * pi r)}, {\apart * (2 * #1 + \t) + \rv * sin(\t * pi r)}
}
\makeatletter
\define@key{DrawFromCenter}{style}[{->}]{
  \tikzset{DrawFromCenterPlane/.style={#1}}
}
\define@key{DrawFromCenter}{r}[1]{
  \def\@R{#1}
}
\define@key{DrawFromCenter}{center}[(0, 0)]{
  \def\@Center{#1}
}
\define@key{DrawFromCenter}{theta}[0]{
  \def\@Theta{#1}
}
\define@key{DrawFromCenter}{phi}[0]{
  \def\@Phi{#1}
}
\presetkeys{DrawFromCenter}{style, r, center, theta, phi}{}
\newcommand*\DrawFromCenter[1][]{
  \setkeys{DrawFromCenter}{#1}{
    \pgfmathsinandcos\sint\cost{\@Theta}
    \pgfmathsinandcos\sinp\cosp{\@Phi}
    \pgfmathsinandcos\sinA\cosA{\angEl}
    \pgfmathsetmacro\DX{\@R*\cost*\cosp}
    \pgfmathsetmacro\DY{\@R*(\cost*\sinp*\sinA+\sint*\cosA)}
    \draw[DrawFromCenterPlane] \@Center -- ++(\DX, \DY);
  }
}
\newcommand*\DrawFromCenterText[2][]{
  \setkeys{DrawFromCenter}{#1}{
    \pgfmathsinandcos\sint\cost{\@Theta}
    \pgfmathsinandcos\sinp\cosp{\@Phi}
    \pgfmathsinandcos\sinA\cosA{\angEl}
    \pgfmathsetmacro\DX{\@R*\cost*\cosp}
    \pgfmathsetmacro\DY{\@R*(\cost*\sinp*\sinA+\sint*\cosA)}
    \draw[DrawFromCenterPlane] \@Center -- ++(\DX, \DY) node {#2};
  }
}
\makeatother
\tikzstyle{snakearrow} = [decorate, decoration={pre length=0.2cm,
  post length=0.2cm, snake, amplitude=.4mm,
  segment length=2mm},thick, ->]
%% document-wide tikz options and styles
\tikzset{%
  >=latex, % option for nice arrows
  inner sep=0pt,%
  outer sep=2pt,%
  mark coordinate/.style={inner sep=0pt,outer sep=0pt,minimum size=3pt,
    fill=black,circle}%
}
\addtolength{\hoffset}{-1.3cm}
\addtolength{\voffset}{-2cm}
\addtolength{\textwidth}{3cm}
\addtolength{\textheight}{2.5cm}
\renewcommand{\footskip}{10pt}
\setlength{\headwidth}{\textwidth}
\setlength{\headsep}{20pt}
\setlength{\marginparwidth}{0pt}
\parindent=0pt
\title{M{\o}lmer-S{\o}rensen gate simulation}

\ifpdf
  % Ensure reproducible output
  \pdfinfoomitdate=1
  \pdfsuppressptexinfo=-1
  \pdftrailerid{}
  \hypersetup{
    pdfcreator={},
    pdfproducer={}
  }
\fi

\begin{document}

\maketitle

\section{Goal}
Derive the expression for simulating and optimizing a M{\o}lmer-S{\o}rensen gate
pulse sequence.

\section{Setup and scope}
We'll discuss a simple two tone pulse sequence where the two tones
are perfectly symmetric around the carrier.
We'll ignore any error in the carrier frequency in this note.
Crosstalk, coupling to carrier and other sideband orders are also ignored.\\

For a typical gate sequence, what we care about are
\begin{enumerate}
\item Ion motion:\\
  The MS interaction will drive each of the motional mode in a spin-dependent way.
  For a proper MS gate, we'd like the final motional state to be identical
  to where we started. Any deviation from this results in a closure error.
\item Spin operation:\\
  The enclosed area in phase-space from the driven motion results in
  a spin-dependent phase which is the main goal of the MS gate.
  Deviation in the control parameter could result in spin/angle error
  in the spin space.
\end{enumerate}

\section{M{\o}lmer-S{\o}rensen interaction}

The effective Hamiltonian for a M{\o}lmer-S{\o}rensen gate sequence can be written as
\eqar{
  H_{MS}=&\frac{\Omega(t)}{2}\sum_{j=1,2}\sum_{k}\eta_{jk}\paren{a_k\ue^{-\ui\theta_k(t)}+a_k^\dagger\ue^{\ui\theta_k(t)}}\sigma^j_x
}
where $j$ is the ion index (simplified to $1$ and $2$)
and $k$ is the motional mode index.
For the ``fixed'' parameters, $\eta_{jk}$ is the Lamb-Dicke parameter for the $j$-th
ion on the $k$-th mode. $a_k$ and $a^\dagger_k$ are the creation and annihilation
operators for the $k$-th mode and the $\sigma_x^j$ is the single qubit spin operator
we are coupling to which we'll set as $x$ in this note.
(The error on the spin axis is ignored.)
For the ``variable'' parameters in the pulse sequence,
$\Omega(t)$ is the time dependent two-photon Rabi frequency (controlled by laser power)
and $\theta_k(t)$ is the time-dependent phase offset between the laser and the $k$-th
mode with,
\eqar{
  \theta_k(t)=&\omega_kt-\theta(t)\\
  =&\omega_kt-\int_0^t\delta(t')\ud t'
}
where $\omega_k$ is the frequency of the $k$-th mode,
$\theta(t)$ is the half the phase difference of the two lasers
and $\delta(t')$ is the (symmetric) detuning of the lasers from the carrier.
(If phase modulation is used, $\theta(t)$ and $\theta_k(t)$
may be discontinuous functions).\\

Using Magnus expansion, we can write down the unitary evalution of the system as
\eqar{
  U_{MS}(\tau)=&\exp\sqr{\sum_{j=1,2}\sum_{k}\frac{\eta_{kj}}{2}\paren{\alpha_k(\tau)a^\dagger_k-\alpha^*_k(\tau)a_k}\sigma^j_x}\exp\paren{\ui\Theta(\tau)\sigma^1_x\sigma^2_x}
}
where
\eqar{
  \alpha_k(\tau)=&\int_0^\tau\Omega(t)\ue^{\ui\theta_k(t)}\ud t
}
describes the displacement of the $k$-th mode, and
\eqar{
  \Theta(\tau)=&\frac{1}{2}\sum_k\eta_{k1}\eta_{k2}\int_0^\tau\!\!\ud t\int_0^t\!\!\ud t'
  \ \Omega(t)\Omega(t')\sin(\theta_k(t)-\theta_k(t'))
}
is the angle of the two-qubit rotation. For a proper MS gate of length $T$,
we need to have $\alpha_k(T)=0$ for all $k$s and the rotation angle
at the end of the pulse $\Theta(T)$ matching the angle we want.\\

For the purpose of optimization, the quantities we care about are.
\begin{enumerate}
\item Closure
  \eqar{
    \alpha_k(T)=&\int_0^T\Omega(t)\ue^{\ui\theta_k(t)}\ud t
  }
\item Gradient of closure w.r.t. mode frequencies
  \eqar{
    \frac{\partial}{\partial\omega_k}\alpha_k(T)=&\int_0^T\frac{\partial}{\partial\omega_k}\Omega(t)\ue^{\ui\theta_k(t)}\ud t\\
    =&\ui\int_0^T\Omega(t)\ue^{\ui\theta_k(t)}\frac{\partial\theta_k(t)}{\partial\omega_k}\ud t\\
    =&\ui\int_0^T\Omega(t)\ue^{\ui\theta_k(t)}t\ud t
  }
  When closure is assumed, this can be re-written as,
  \eqar{
    \frac{\partial}{\partial\omega_k}\alpha_k(T)=&\ui\int_0^T\Omega(t)\ue^{\ui\theta_k(t)}t\ud t\\
    =&\ui\left.t\int_0^t\Omega(t')\ue^{\ui\theta_k(t')}\ud t'\right|_0^T-\ui\int_0^T\ud t\int_0^t\Omega(t')\ue^{\ui\theta_k(t')}\ud t'\\
    =&-\ui\int_0^T\ud t\int_0^t\Omega(t')\ue^{\ui\theta_k(t')}\ud t'
  }
  which is proportional to average displacement.
  Moreover, if the pulse is symmetric, zeroing this value will also automatically
  zero the final displacement thus remove the need to optimize
  two values at the same time.
\item Enclosed area
  \eqar{
    A_k=&\int_0^T\!\!\ud t\int_0^t\!\!\ud t'
    \ \Omega(t)\Omega(t')\sin(\theta_k(t)-\theta_k(t'))\\
    =&\int_0^T\!\!\ud t\int_0^t\!\!\ud t'
    \ \Omega(t)\Omega(t')\sin(\omega_k(t-t')-(\theta(t)-\theta(t')))
  }
\item Gradient of enclosed area w.r.t. mode frequencies
  \eqar{
    \frac{\partial}{\partial\omega_k}A_k=&\int_0^T\!\!\ud t\int_0^t\!\!\ud t'
    \ \Omega(t)\Omega(t')\sin(\theta_k(t)-\theta_k(t'))\\
    =&\int_0^T\!\!\ud t\int_0^t\!\!\ud t'
    \ \Omega(t)\Omega(t')\sin(\omega_k(t-t')-(\theta(t)-\theta(t')))
  }
\end{enumerate}

\end{document}
